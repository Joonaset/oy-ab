\documentclass{article}
\usepackage[utf8]{inputenc}

\author{Ryhmä 5}
\title{Verkkosivut yritykselle\\
\large Suunnitteludokumentti}


\begin{document}
\maketitle
\section{Johdanto}
Projektimme tavoitteena on suunnitella ja toteuttaa yriteykselle konseptimalli verkkosivustosta, joka sisältää verkkokaupan. Pääpaino on verkkokaupan
toiminnoissa, koska se vaatii eniten osaamista koko fullstack-kirjolla. Projekti voidaan jakaa osiin, ohjelmointikielien mukaan. HTML määrittelee verkkosivun rakenteen,
CSS ulkoasun, JavaScript toiminnallisuudet käyttäjälaitteen puolella, PHP palvelimen toiminnallisuudet ja SQL tietokannan.

\section{Määrittely}
\subsection{HTML}
Verkkosivusto rakentuu kolmesta eri sivusta, etusivusta, joka sisältää myös verkkokaupan, yrityksen infosivusta ja yhteystietosivusta. Kullekin sivulle yhteisiä elementtejä ovat yläosan navigaatiopalkki, sekä alaosan infolaatikko.

\subsection{CSS}
Koko sivusto käyttää yhtä CSS-tiedostoa, jolla saadaan yhtenäinen, sivustonlaajuinen ulkoasu.

\subsection{JavaScript}
JavaScript kuuntelee asiakkaan komentoja ja käsittelee ne, esimerkiksi lähettäen kutsun palvelimelle, tai päivittämällä sivun hakuehtojen mukaiseksi.
\subsection{PHP}
PHP käsittelee asiakkaan pyynnöt, tekee palvelinpuolen tietokantakutsut ja palauttaa ne asiakkaalle.
\subsection{SQL}
SQL-tietokanta sisältää verkkokaupan tuotteet, rekisteröidyt asiakkaat ja tilaukset.

\section{Asiakaskokemukset}

\begin{itemize}
    \item Asiakas voi etsiä tuotetta hakusanalla, kategorialla tai tuotetunnuksella.
    \item Asiakas voi järjestää hakutulokset aakkosittain tai hinnan mukaan.
\end{itemize}

\end{document}



